\chapter{Introduction}

Model-View-Controller, or {\em MVC} for short, is a common pattern in Web frameworks 
where it is used predominantly to build HTML applications. The {\em model} refers to the 
application's data, the {\em view} to the application's data presentation and the 
{\em controller} to the part of the system responsible for managing input, updating models 
and producing output.

Web UI frameworks can be categorized as {\em action-based} or {\em component-based}. In an action-based 
framework, HTTP requests are routed to controllers where they are turned into actions by application code; 
in a component-based framework, HTTP requests are grouped and typically handled by framework components
with little or no interaction from application code. In other words, in a component-based framework,
the majority of the controller logic is provided by the framework instead of the application.

The API defined by this specification falls into the action-based 
category and is, therefore, not intended to be a replacement for component-based framworks such as
JavaServer Faces (JSF) \cite{jsf}, but simply a different approach to building Web applications on the 
Java EE platform.

\section{Goals}
\label{goals}

The following are goals of the API:

\begin{description}
\item[Goal 1] Leverage existing Java EE technologies. 
\item[Goal 2] Integrate with CDI \cite{cdi} and Bean Validation \cite{bv11}.
\item[Goal 3] Define a solid core to build \mvc\ applications without necessarily supporting all the 
features in its first version.
\item[Goal 4] Explore layering on top of \jaxrs\ for the purpose of re-using its
matching and binding layers.
\item[Goal 5] Provide built-in support for JSPs and Facelets view languages.
\end{description}

\section{Non-Goals}
\label{non_goals}

The following are non-goals of the API:

\begin{description}
\item[Non-Goal 1] Define a new view (template) language and processor.
\item[Non-Goal 2] Support standalone implementations of \mvc\ running outside of Java EE.
\item[Non-Goal 3] Support REST services not based on \jaxrs.
\item[Non-Goal 4] Provide built-in support for view languages that are not part of Java EE.
\end{description}

It is worth noting that, even though a standalone implementation of \mvc\ that runs outside
of Java EE is a non-goal, this specification shall not intentionally prevent implementations
to run in other environments, provided that those environments include support all the 
EE technologies required by \mvc.

\section{Additional Information}
\label{additional_information}

The issue tracking system for this release can be found at:

\begin{quote}
https://java.net/jira/browse/MVC\_SPEC
\end{quote}

The corresponding Javadocs can be found online at:

\begin{quote}
https://mvc-spec.java.net/
\end{quote}

The reference implementation can be obtained from:

\begin{quote}
https://ozark.java.net/
\end{quote}

The expert group seeks feedback from the community on any aspect of this specification, please 
send comments to:

\begin{quote}
users@mvc-spec.java.net
\end{quote}

\section{Terminology}
\label{terminology}

Most of the terminology used in this specification is borrowed from other specifications 
such as \jaxrs\ and CDI. We use the terms {\em per-request} and {\em request scoped} as
well as {\em per-application} and {\em application scoped} interchangeably.

\section{Conventions}

The keywords `MUST', `MUST NOT', `REQUIRED', `SHALL', `SHALL NOT', `SHOULD', `SHOULD NOT', 
`RECOMMENDED', `MAY', and `OPTIONAL' in this document are to be interpreted as described in 
RFC 2119\cite{rfc2119}. 

Assertions defined by this specification are formatted as {\textbf{\llb{an-assertion}\rrb}} 
using a descriptive name as the label and are all listed in Appendix \ref{assertions}.

Java code and sample data fragments are formatted as shown in figure \ref{ex1}:

\begin{figure}[hbtp]
\caption{Example Java Code}
\label{ex1}
\begin{listing}{1}
package com.example.hello;

public class Hello {
    public static void main(String args[]) {
        System.out.println("Hello World");
    }
}\end{listing}
\end{figure}

URIs of the general form `http://example.org/...' and `http://example.com/...' represent application 
or context-dependent URIs.

All parts of this specification are normative, with the exception of examples, notes and sections
explicitly marked as `Non-Normative'. Non-normative notes are formatted as shown below.

\begin{nnnote*}
This is a note.
\end{nnnote*}

\section{Expert Group Members} 
\label{expert_group}

This specification is being developed as part of JSR 371 under the Java Community Process. The following are the present expert
group members:

\begin{list}{$-$}{\parsep 0em \labelwidth 0em}
\item Mathieu Ancelin (Individual Member)
\item Ivar Grimstad (Individual Member)
\item Neil Griffin (Liferay, Inc)
\item Joshua Wilson (RedHat)
\item Rodrigo Turini (Caelum)
\item Stefan Tilkov (innoQ Deutschland GmbH)
\item Guilherme de Azevedo Silveira (Individual Member)
\item Frank Caputo (Individual Member)
\item Christian Kaltepoth (Individual Member)
\item Woong-ki Lee (TmaxSoft, Inc.)
\item Paul Nicolucci (IBM)
\item Kito D. Mann (Individual Member)
\end{list}

\section{Acknowledgements}
\label{acks}

During the course of this JSR we received many excellent suggestions. Special thanks to Marek
Potociar, Dhiru Pandey and Ed Burns, all from Oracle. In addition, to everyone in the user's alias
that followed the expert discussions and provided feedback.

