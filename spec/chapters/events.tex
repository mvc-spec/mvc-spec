\chapter{Events}
\label{events}

This chapter introduces a mechanism by which MVC applications can be informed 
of important events that occur while processing a request. This mechanism 
is based on CDI events that can be fired by implementations and observed by 
applications.

\section{Observers}
\label{observers}

The package \code{javax.mvc.event} defines a number of event types that MUST be 
fired by implementations during the processing of a request \assertref{event-firing}. 
Implementations MAY
extend this set and also provide additional information on any of the events defined
by this specification. The reader is referred to the implementation's documentation
for more information on event support.

Observing events can be useful for applications to learn about the lifecycle of a 
request, provide application-level logging, monitor performance, etc. Chapter 
\ref{view_engines} describes the algorithm used by implementations to select a
specific view engine for processing. This information is made available to any
application (or framework) that observes the \code{ViewEngineSelectedEvent}. For
example,

\begin{listing}{1}
@ApplicationScoped
public class EventObserver {

    public void onViewEngineSelected(@Observes ViewEngineSelectedEvent e) {
        ...
    }
}
\end{listing}

Observer methods in CDI are defined using the \code{@Observes} annotation on 
a parameter position.
The class \code{EventObserver} is a CDI bean in application scope whose method
\code{onViewEngineSelected} is called every time a \code{ViewEngineSelectedEvent}
is fired by the implementation. 

To complete the example, let us assume that the information about the selected
view engine needs to be conveyed to the client. To ensure that this information
is available to a view returned to the client, the \code{EventObserver} class
can inject and update the same request-scope bean accessed by the view:

\begin{listing}{1}
@ApplicationScoped
public class EventObserver {

    @Inject
    private EventBean eventBean;

    public void onViewEngineSelected(@Observes ViewEngineSelectedEvent e) {
        eventBean.setView(e.getView());
        eventBean.setEngine(e.getEngine());
    }
}
\end{listing}

For more information about the interaction between views and models, the reader
is referred to Section~\ref{models}. 

Events fired by implementations are synchronous,
so it is recommended that applications carry out only simple tasks in 
their observer methods, avoiding long-running computations as well as blocking calls.

\begin{ednote}
Synchronous vs.~asynchronous event processing should be reviewed 
based on any new capabilities available in CDI 2.0.
\end{ednote}
