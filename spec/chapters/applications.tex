\chapter{Applications}
\label{applications}

This chapter introduces the notion of an \mvc\ application and explains how it 
relates to a \jaxrs\ application.

\section{\mvc\ Applications}
\label{mvc_applications}

An \mvc\ application consists of one or more \jaxrs\ resources that are annotated with 
\Controller\ and, just like \jaxrs\ applications, zero or more providers. If no
resources are annotated with \Controller, then the resulting application is a \jaxrs\
application instead. In general, everything that applies to a \jaxrs\ application
also applies to an \mvc\ application. Some \mvc\ applications may be {\em hybrid}
and include a mix of \mvc\ controllers and \jaxrs\ resource methods.

The controllers and providers that make up an application are configured via an 
application-supplied subclass of \code{Application} from \jaxrs. An implementation 
MAY provide alternate mechanisms for locating controllers, but as in \jaxrs, 
the use of an \code{Application} subclass is the only way to guarantee portability.

All the rules described in the Servlet section of the \jaxrs\ Specification
\cite{jaxrs} apply to \mvc\ as well. This section recommends the use of
the Servlet 3 framework pluggability mechanism and describes its semantics
for the cases in which an \code{Application} subclass is present and absent. 

The path in the application's URL space in which \mvc\ controllers live must
be specified either using the \ApplicationPath\ annotation on the application
subclass or in the web.xml as part of the \code{url-pattern} element. \mvc\
applications SHOULD use a non-empty path or pattern: i.e., {\em "/"} or {\em "/*"} 
should be avoided whenever possible. 

The reason for this is that \mvc\ implementations
often forward requests to the Servlet container, and the use of the 
aforementioned
values may result in the unwanted processing of the forwarded request
by the \jaxrs\ servlet once again. Most \jaxrs\ applications avoid using
these values, and many use \code{"/resources"} or \code{"/resources/*"} by 
convention. For consistency, it is recommended for MVC applications to use
these patterns as well.

\section{MVC Context}
\label{mvc_context}

MVC applications can inject an instance of {\tt MvcContext} to access configuration,
security and path-related information. Instances of {\tt MvcContext} are provided
by implementations and are always in application scope \assertref{mvc-context}. 
For convenience, the
{\tt MvcContext} instance is also available using the name {\tt mvc} in EL.

As an example, a view can refer to a CSS file by using the context path available in 
the {\tt MvcContext} object as follows:

\begin{listing}{1}
<link rel="stylesheet" type="text/css" href="${mvc.contextPath}/my.css">
\end{listing}


For more information on security see Chapter~\ref{security}; for more information 
about the {\tt MvcContext} in general, refer to the Javadoc for the type.

\section{Providers in \mvc}
\label{providers_in_mvc}

Implementations are free use their own providers in order to modify the standard 
\jaxrs\ pipeline for the purpose of implementing the \mvc\ semantics. Whenever mixing 
implementation and application providers, care should be taken to ensure the correct 
execution order using priorities.

\section{Annotation Inheritance}
\label{annotation_inheritance}

\mvc\  applications MUST follow the annotation inheritance rules defined by \jaxrs\ 
\assertref{annotation-inheritance}. Namely, \mvc\ annotations may be used on methods of a 
super-class or an implemented interface. Such annotations are inherited 
by a corresponding sub-class or implementation class method provided that the method does 
not have any \mvc\ or \jaxrs\ annotations of its own: 
i.e., if a subclass or implementation method has any \mvc\ or \jaxrs\ annotations then all of the
annotations on the superclass or interface method are ignored.

Annotations on a super-class take precedence over those on an implemented interface. The
precedence over conflicting annotations defined in multiple implemented interfaces is implementation dependent.
Note that, in accordance to the \jaxrs\ rules, inheritance of class or interface annotations 
is not supported. 
